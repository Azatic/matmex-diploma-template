% Год, город, название университета и факультета предопределены,
% но можно и поменять.
% Если англоязычная титульная страница не нужна, то ее можно просто удалить.
\filltitle{ru}{
    chair              = {Кафедра, на которой работает научник},
    title              = {Шаблон отчёта по учебной практике},
    group              = {ХХБ.ХХ-мм},
    % Здесь указывается тип работы. Возможные значения:
    %   practice - Отчёт по учебной практике
    %   diploma - Отчёт по преддипломной практике
    %   master - ВКР магистра
    %   bachelor - ВКР бакалавра
    type               = {practice},
    position           = {},
    author             = {Имя Отчество Фамилия},
    supervisorPosition = {проф. каф. СП, д.ф.-м.н., проф.},
    supervisor         = {Н.Н. Научник},
    % Если консультанта нет, закомментировать или удалить вовсе
    reviewerPosition   = {должность ООО <<Место работы>> степень},
    reviewer           = {К.К. Консультант}
}
% \filltitle{en}{
%     chair              = {Department of Software Engineering},
%     title              = {xxx},
%     author             = {xxx},
%     supervisorPosition = {xxx},
%     supervisor         = {xxxx},
%     reviewerPosition   = {Software Developer at IntelliJ Labs Co. Ltd.},
%     reviewer           = {xxx},
% }

