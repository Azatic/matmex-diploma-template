% !TeX spellcheck = ru_RU
% !TEX root = vkr.tex

\section{Эксперимент (желательно к Новому году)}
Как мы проверяем, что  всё удачно получилось.  К Новому году для промежуточного отчета желательно хотя бы описать как он будет прово\-диться и на чем.

\subsection{Условия эксперимента}
Железо (если актуально);  версии ОС, компиляторов и параметры командной строки; почему мы выбрали именно эти тесты; входные дан\-ные, на которых проверяем наш подход, и почему мы выбрали именно их.

\subsection{Исследовательские вопросы }
По-английски называется \emph{research questions}, в тексте можно ссылаться на них как RQ1, RQ2, и т.~д.
Необходимо сформулировать, чего мы хотели бы добиться работой (2 пункта будет хорошо):

\begin{itemize}
\item Хотим алгоритм, который лучше вот таких-то остальных.
\item Если в подходе можно включать/выключать составляющие, то насколько существенно каждая составляющая влияет на улучшения.
\item Если у нас строится приближение каких-то штук, то на сколько точными будут эти приближения.
\item и т.п.
\end{itemize}

Иногда в работах это называют гипотезами, которые потом проверяют. Далее в тексте можно ссылаться на research questions как \textsc{RQ}, это обще\-при\-нятое сокращение.

\subsection{Метрики}

Как мы сравниваем, что результаты двух подходов лучше или хуже:
\begin{itemize}
\item Производительность.
\item Строчки кода.
\item Как часто алгоритм \enquote{угадывает} правильную классификацию входа.
\end{itemize}

\noindent Иногда метрики вырожденные (да/нет), это не очень хорошо, но если в области исследований так принято, то ладно.

\subsection{Результаты}
Результаты понятно что такое. Тут всякие таблицы и графики, как в таблице \ref{time_cmp_obj_func}. Обратите внимание, как цифры выровнены по правому краю, названия по центру, а разделители $\times$ и $\pm$ друг под другом.

Скорее всего Ваши измерения будут удовлетворять нормальному распределению, в идеале это надо проверять с помощью критерия Кол\-могорова и т.п.
Если критерий этого не подтверждает, то у Вас что-то сильно не так с измерениями, надо проверять кэши процессора, отключать Интернет во время измерений, подкручивать среду исполне\-ния (англ. runtime), что\-бы сборка мусора не вмешивалась и т.п.
Если критерий удовлетворён, то необходимо либо указать мат. ожидание и доверительный/предсказы\-вающий интервал, либо написать, что все измерения проводились с погрешностью, например, в 5\%.
Замечание: если у вас получится улуч\-шение производительности в пределах погреш\-ности, то это обязательно вызовет вопросы.

В этом разделе надо также коснуться Research Questions.

\subsubsection{RQ1} Пояснения
\subsubsection{RQ2} Пояснения

\begin{table}
\def\arraystretch{1.1}  % Растяжение строк в таблицах
\setlength\tabcolsep{0.2em}
\centering
% \resizebox{\linewidth}{!}{%
    \caption{Производительность какого-то алгоритма при различных разрешениях картинок  (меньше --- лучше), в мс.,  CI=0.95. За пример таблички кидаем чепчики в честь Я.~Кириленко}
    \begin{tabular}[C]{
    |S[table-format=4.4,output-decimal-marker=\times]
    *4{|S
          [table-figures-uncertainty=2, separate-uncertainty=true, table-align-uncertainty=true,
          table-figures-integer=3, table-figures-decimal=2, round-precision=2,
          table-number-alignment=center]
          }
    |}
    \toprule
        \multicolumn{1}{|c|}{Resolution} & \multicolumn{1}{c|}{\textsc{TENG}} & \multicolumn{1}{c|}{\textsc{LAPM}} &
        \multicolumn{1}{|c|}{\textsc{VOLL4}} \\ \hline
        1920.1080 & 406.23 \pm 0.94 & 134.06 \pm 0.35 & 207.45 \pm 0.42  \\ \hline
        1024.768  & 145.0 \pm 0.47  & 39.68 \pm 0.1   &  52.79  \pm 0.1 \\ \hline
        464.848   & 70.57 \pm 0.2   & 19.86 \pm 0.01     & 32.75  \pm 0.04 \\ \hline
        640.480   & 51.10 \pm 0.2   & 14.70 \pm 0.1 & 24  \pm 0.04 \\ \hline
        160.120   & 2.4 \pm 0.02    & 0.67 \pm 0.01      & 0.92  \pm 0.01 \\
        \bottomrule
    \end{tabular}%
%}
    \label{time_cmp_obj_func}
\end{table}

\clearpage
% !TeX spellcheck = ru_RU
% !TEX root = vkr.tex

\newcolumntype{C}{ >{\centering\arraybackslash} m{4cm} }
\newcommand\myvert[1]{\rotatebox[origin=c]{90}{#1}}
\newcommand\myvertcell[1]{\multirowcell{5}{\myvert{#1}}}
\newcommand\myvertcelll[1]{\multirowcell{4}{\myvert{#1}}}
\newcommand\myvertcellN[2]{\multirowcell{#1}{\myvert{#2}}}


\afterpage{%
    \clearpage% Flush earlier floats (otherwise order might not be correct)
    \thispagestyle{empty}% empty page style (?)
    \begin{landscape}% Landscape page
        \centering % Center table

\begin{tabular}{|c|c|c|c|c|c|c|c|c|c|c|c|c|c|c|c|c|c|}\hline
%& \multicolumn{17}{c|}{} \\ \hline 
\multirowcell{2}{Код модуля \\в составе \\ дисциплины,\\практики и т.п. }
  &\myvertcellN{2}{Трудоёмкость}
  & \multicolumn{10}{c|}{\tiny{Контактная работа обучающихся с преподавателем}} 
  & \multicolumn{5}{c|}{\tiny{Самостоятельная работа}} 
  & \myvertcellN{2}{\tiny Объем активных и интерактивных } 
  \\ \cline{3-17}

&& \myvertcellN{2}{лекции} 
    &\myvertcellN{2}{семинары}
    &\myvertcellN{2}{консультации}
    &\myvertcellN{2}{\small практические  занятия }
    &\myvertcellN{2}{\small лабораторные работы }
    &\myvertcellN{2}{\small контрольные работы }
    &\myvertcellN{2}{\small коллоквиумы }
    &\myvertcellN{2}{\small текущий контроль }
    &\myvertcellN{2}{\small промежуточная аттестация }
    &\myvertcellN{2}{\small итоговая аттестация }
    
    &\myvertcellN{2}{\tiny под руководством    преподавателя}
    &\myvertcellN{2}{\tiny в присутствии     преподавателя   }
    &\myvertcellN{2}{\tiny с использованием    методических}
    &\myvertcellN{2}{\small текущий контроль}
    &\myvertcellN{2}{\tiny промежуточная аттестация } 
    &     \\
&& &&&&&&&&& &&&&&&\\
&& &&&&&&&&& &&&&&&\\
&& &&&&&&&&& &&&&&&\\ 
&&&&&&&&&&& &&&&&&\\ 
&&&&&&&&&&& &&&&&&\\ 
&&&&&&&&&&& &&&&&&\\ \hline
Семестр 3 & 2 &30  &&&&&&&&2   & &&&18 &&20 &10\\ \hline
          &   &2-42&&&&&&&&2-25& &&&1-1&&1-1&\\ \hline
Итого     & 2 &30  &&&&&&&&2   & &&&18 &&20 &10\\ \hline
\end{tabular}

        \captionof{table}{Если таблица очень большая, то можно её изобразить на отдельной портретной странице}
    \end{landscape}
    \clearpage% Flush page
}



\subsection{Обсуждение результатов}

Чуть более неформальное обсуждение, то, что сделано. Например, почему метод работает лучше остальных? Или, что делать со случаями, когда метод классифицирует вход некорректно.
