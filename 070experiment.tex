% !TeX spellcheck = ru_RU

Как мы проверяем, что  всё удачно получилось

\subsection{Условия эксперимента}
Железо (если актуально); входные данные, на которых проверяем наш подход; почему мы выбрали именно эти тесты

\subsection{Исследовательские вопросы (Research questions)}
Надо сформулировать то, чего мы хотели бы добиться работой (2 штуки будет хорошо):

\begin{itemize}
\item Хотим алгоритм, который лучше вот таких-то остальных
\item Если в подходе можно включать/выключать составляющие, то насколько существенно каждая составляющая влияет на улучшения
\item Если у нас строится приближение каких-то штук, то на сколько точными будут эти приближения
\item и т.п.
\end{itemize}

\subsection{Метрики}

Как мы сравниваем, что результаты двух подходов лучше или хуже
\begin{itemize}
\item Производительность
\item Строчки кода
\item Как часто алгоритм "угадывает" правильную классификацию входа
\end{itemize}

Иногда метрики вырожденные (да/нет), это не очень хорошо, но если в области исследований так принято, то ладно.

\subsection{Результаты}
Результаты понятно что такое. Тут всякие таблицы и графики

В этом разделе надо также коснуться Research Questions.

\subsubsection{RQ1} Пояснения
\subsubsection{RQ2} Пояснения

\clearpage
% !TeX spellcheck = ru_RU
% !TEX root = vkr.tex

\newcolumntype{C}{ >{\centering\arraybackslash} m{4cm} }
\newcommand\myvert[1]{\rotatebox[origin=c]{90}{#1}}
\newcommand\myvertcell[1]{\multirowcell{5}{\myvert{#1}}}
\newcommand\myvertcelll[1]{\multirowcell{4}{\myvert{#1}}}
\newcommand\myvertcellN[2]{\multirowcell{#1}{\myvert{#2}}}


\afterpage{%
    \clearpage% Flush earlier floats (otherwise order might not be correct)
    \thispagestyle{empty}% empty page style (?)
    \begin{landscape}% Landscape page
        \centering % Center table

\begin{tabular}{|c|c|c|c|c|c|c|c|c|c|c|c|c|c|c|c|c|c|}\hline
%& \multicolumn{17}{c|}{} \\ \hline 
\multirowcell{2}{Код модуля \\в составе \\ дисциплины,\\практики и т.п. }
  &\myvertcellN{2}{Трудоёмкость}
  & \multicolumn{10}{c|}{\tiny{Контактная работа обучающихся с преподавателем}} 
  & \multicolumn{5}{c|}{\tiny{Самостоятельная работа}} 
  & \myvertcellN{2}{\tiny Объем активных и интерактивных } 
  \\ \cline{3-17}

&& \myvertcellN{2}{лекции} 
    &\myvertcellN{2}{семинары}
    &\myvertcellN{2}{консультации}
    &\myvertcellN{2}{\small практические  занятия }
    &\myvertcellN{2}{\small лабораторные работы }
    &\myvertcellN{2}{\small контрольные работы }
    &\myvertcellN{2}{\small коллоквиумы }
    &\myvertcellN{2}{\small текущий контроль }
    &\myvertcellN{2}{\small промежуточная аттестация }
    &\myvertcellN{2}{\small итоговая аттестация }
    
    &\myvertcellN{2}{\tiny под руководством    преподавателя}
    &\myvertcellN{2}{\tiny в присутствии     преподавателя   }
    &\myvertcellN{2}{\tiny с использованием    методических}
    &\myvertcellN{2}{\small текущий контроль}
    &\myvertcellN{2}{\tiny промежуточная аттестация } 
    &     \\
&& &&&&&&&&& &&&&&&\\
&& &&&&&&&&& &&&&&&\\
&& &&&&&&&&& &&&&&&\\ 
&&&&&&&&&&& &&&&&&\\ 
&&&&&&&&&&& &&&&&&\\ 
&&&&&&&&&&& &&&&&&\\ \hline
Семестр 3 & 2 &30  &&&&&&&&2   & &&&18 &&20 &10\\ \hline
          &   &2-42&&&&&&&&2-25& &&&1-1&&1-1&\\ \hline
Итого     & 2 &30  &&&&&&&&2   & &&&18 &&20 &10\\ \hline
\end{tabular}

        \captionof{table}{Если таблица очень большая, то можно её изобразить на отдельной портретной странице}
    \end{landscape}
    \clearpage% Flush page
}



\subsection{Обсуждение результатов}

Чуть более неформальное обсуждение, то, что сделано. Например, почему метод работает лучше остальных? Или, что делать со случаями, когда метод классифицирует вход некорректно. 

