% !TeX spellcheck = ru_RU
% !TEX root = vkr.tex

Реализация в широком смысле: что таки было сделано. Скорее всего самый большой раздел.

\emph{Крайне желательно} к Новому году иметь что-то, что сюда можно написать.

Для понимания того как курсовая записка (отзыв по учебной практике/ВКР) должна писаться, можно посмотреть видео ниже. Они про научные доклады и написание научных статей, учебные практики и ВКР отличаются тем, что тут есть требования отдельных глав (слайдов) со списком задач и списком результатов. Но даже если вы забьёте не требования специфичные для ВКР, и соблюдете все рекомендации ниже, получившиеся скорее всего будет лучше чем первая попытка 99\% ваших однокурсников.

\begin{enumerate}
\item <<Как сделать великолепный научный доклад>> от Саймона Пейтона Джонса~\cite{SPJGreatTalk} (на английском)
\item <<Как написать великолепную научную статью>> от Саймона Пейтона Джонса~\cite{SPJGreatPaper} (на английском)
\item <<Как писать статьи так, чтобы люди их смогли прочитать>> от Дэрэка Драйера~\cite{DreyerYoutube2020} (на английском). По предыдующей ссылке разбираются, что должно быть в статье, т.е. как она должна выглядеть на высоком уровне. Тут более низкоуровнево изложено как должны быть устроены параграфы и т.п.
\item Ещё можно посмотреть <<How to Design Talks>>~\cite{JhalaYoutube2020} от Ranjit Jhala, но я думаю, что первых трех ссылок всем хватит.
\end{enumerate}


\subsection{Основные грабли}
Здесь мы будем собирать основные ошибки, которые случаются при написании текстов. В интернетах тоже можно найти коллекции типичных косяков\footnote{\href{https://www.read.seas.harvard.edu/~kohler/latex.html}{https://www.read.seas.harvard.edu/~kohler/latex.html}}.


\begin{lstlisting}[caption=Название обязательно, language=Caml, frame=single]
let x = 5 in x+1
\end{lstlisting}

Рекомендуется использовать красивые греческие буквы $\Phi,\phi$ по-умол\-ча\-нию, а именно $\phi$ вместо $\varphi$. В данном шаблоне команды для этих букв переставлены местами по сравнению с ванильным \TeX'ом.


\subsection{Выделения куска листинга с помощью tikz}
\url{https://tex.stackexchange.com/questions/284311}

\begin{lstlisting}[escapechar=!,basicstyle=\ttfamily]
#include <stdio.h>
#include <math.h>

int main () {
  double c=-1;
  double z=0;
  int i;

  printf (``For c = %lf:\n'', c );
  for ( i=0; i<10; i++ ) {
    printf ( !\tikzmark{a}!"z %d = %lf\n"!\tikzmark{b}!, i, z );
    z = pow(z,2) + c;
  }
}
\end{lstlisting}

\begin{tikzpicture}[use tikzmark]
\draw[fill=gray,opacity=0.1]
  ([shift={(-3pt,2ex)}]pic cs:a) 
    rectangle 
  ([shift={(3pt,-0.65ex)}]pic cs:b);
\end{tikzpicture}
