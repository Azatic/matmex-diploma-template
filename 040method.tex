% !TeX spellcheck = ru_RU
% !TEX root = vkr.tex

Реализация в широком смысле: что таки было сделано. Скорее всего самый большой раздел.

\emph{Крайне желательно} к Новому году иметь что-то, что сюда можно написать.

Для понимания того как курсовая записка (отзыв по учебной практике/ВКР) должна писаться, можно посмотреть видео ниже. Они про научные доклады и написание научных статей, учебные практики и ВКР отличаются тем, что тут есть требования отдельных глав (слайдов) со списком задач и списком результатов. Но даже если вы забьёте не требования специфичные для ВКР, и соблюдете все рекомендации ниже, получившиеся скорее всего будет лучше чем первая попытка 99\% ваших однокурсников.

\begin{enumerate}
\item ``Как сделать великолепный научный доклад'' от Саймона Пейтона Джонса~\cite{SPJGreatTalk} (на английском).
\item ``Как написать великолепную научную статью'' от Саймона Пейтона Джонса~\cite{SPJGreatPaper} (на английском).
\item ``Как писать статьи так, чтобы люди их смогли прочитать'' от Дэрэка Драйера~\cite{DreyerYoutube2020} (на английском). По предыдующей ссылке разбираются, что должно быть в статье, т.е. как она должна выглядеть на высоком уровне. Тут более низкоуровнево изложено как должны быть устроены параграфы и т.п.
\item Ещё можно посмотреть ``How to Design Talks''~\cite{JhalaYoutube2020} от Ranjit Jhala, но я думаю, что первых трех ссылок всем хватит.
\end{enumerate}


\subsection{Некоторые типичные ошибки}
Здесь мы будем собирать основные ошибки, которые случаются при написании текстов. В интернетах тоже можно найти коллекции типичных косяков\footnote{\href{https://www.read.seas.harvard.edu/~kohler/latex.html}{https://www.read.seas.harvard.edu/~kohler/latex.html} (дата доступа:   \DTMdate{2022-12-16}).}.

Рекомендуется по-умол\-ча\-нию использовать красивые греческие буквы $\sigma$  и $\phi$, а именно $\phi$ вместо $\varphi$. В данном шаблоне команды для этих букв переставлены местами по сравнению с ванильным \TeX'ом.

Также, если работа пишется на русском языке, необходимо, чтобы работа была написана на \textit{грамотном} русском языке (даже, если автор, например, казах). Это включает в себя:
\begin{itemize}
  \item разделы должны оформляться с помощью \verb=\section{...}=, а также \verb=\subsection= и т.~п.; не нужно пытаться нумеровать вручную;
  \item точки после окончания предложений;
  \item пробелы после запятых  и точек, в конце слова перед скобкой;
  \item неразрывные пробелы, там, где нужны пробелы, но переносить на другую строку нельзя, например \verb=т.~е.= или \verb=А.~Н.~Терехов=;
  \item дефис, там где нужен дефис (обозначается с помощью одиночного ``минуса'' (англ. dash) на клавиатуре);
  \item двойной дефис, там где он нужен; а именно  при указании промежутка в цифрах: в 1900--1910 г.~г., стр. 150--154;
  \item тире (т.~е. \verb=---= --- тройной минус) на месте тире, а не что-то другое.
  \item правильные двойные кавычки, а именно они должны начинаться с двойного backtick \verb=``= (кнопка с буквой Ё), а заканчиваться двойной одинарной кавычкой \verb=''= (кнопка с буквой Э);
  \item все перечисления должны оформляться с помощью \verb=\enumerate= или \verb=\itemize=; пункты перечислений должны либо начинаться с заглавной буквой и заканчиваться точкой, либо начинаться со строчной и закачиваться точкой с запятой; последний пункт перечислений всегда заканчивается точкой.
\end{itemize}



\subsection{Листинги, картинка и прочий ``не текст''}

Различный ``не текст'' имеет свойство отображаться не там, где он написан в текстовом в \LaTeX{}, поэтому у него должна быть самодостаточная понятная подпись \verb=\caption{...}=, уникальная метка \verb=\label{...}=, чтобы на неё можно было бы ссылаться в тексте с помощью \verb=\ref{...}=. Ниже вы сможете таблицу \ref{time_cmp_obj_func}, на которую мы сослались буквально только что.


\begin{lstlisting}[caption={Название для листинга кода. Достаточно длинное, чтобы люди, которые смотрят картинку сразу после названия статьи (т.~е. все люди), смогли разобраться и понять к чему в статье листинги, картинки и прочий ``не текст''.}, language=Caml, frame=single]
  let x = 5 in x+1
\end{lstlisting}



\subsubsection{Выделения куска листинга с помощью tikz}
Это обывает полезно в текста, а ещё чаще --- в презентациях. Пример сделан на основе вопроса на \textsc{StackOveflow}\footnote{\url{https://tex.stackexchange.com/questions/284311} (дата доступа:   \DTMdate{2022-12-16}).}.

\begin{figure}
\begin{lstlisting}[escapechar=!,basicstyle=\ttfamily, language=c]
#include <stdio.h>
#include <math.h>

int main(void)
{
  double c = -1;
  double z = 0;

  printf ("For c = %lf:\n", c);
  for (int i=0; i<10; i++ ) {
    printf ( !\tikzmark{a}!"z %d = %lf\n"!\tikzmark{b}!, i, z);
    z = pow(z, 2) + c;
  }
}
\end{lstlisting}

\begin{tikzpicture}[use tikzmark]
\draw[fill=gray,opacity=0.1]
  ([shift={(-3pt,2ex)}]pic cs:a)
    rectangle
  ([shift={(3pt,-0.65ex)}]pic cs:b);
\end{tikzpicture}
\caption{Пример листинга и \textsc{TIKZ} декорации к нему, оформленные в окружении \texttt{figure}. Обратите внимание, что рисунок отображается не там, где он в документе, а может ``плавать''.}
\end{figure}