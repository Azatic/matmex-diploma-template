% !TeX spellcheck = ru_RU
% !TEX root = vkr.tex
% Основные необходимые пакеты и команды.

% TODO: Основные пакеты должны быть в класс-файле?
%%% Пакеты
%% Графика
\usepackage[table]{xcolor} % Работа с цветами
\usepackage{graphicx} % Вставка графики в текст (изображения, в т.ч. pdf)
\graphicspath{ {figures/} } % Задает папку по-умолчанию для графики

\usepackage{wrapfig2} % Позволяет вставлять графику, обтекаемую текстом
\usepackage{caption} % Настройка подписей "не текста"
\usepackage{subcaption} % Подписи для разделенного "не текста"

%% Математика
\usepackage{amsmath, amsfonts, amssymb, amsthm, mathtools} % "Адекватная" работа с математикой в LaTeX
\usepackage{thmtools} % Мощный пакет для работы с математическими окружениями типа "теорема"
\usepackage{thm-restate} % Дополнение к предыдущему пакету, позволяющее повторять теоремы

%% Таблицы
\usepackage{tabularx} % Добавляет столбец типа "X", который автоматически занимает максимально возможное место
\usepackage{makecell} % Позволяет кастомизировать ячейки таблицы

%% Код
\usepackage{listings} % Позволяет вставлять код в документ
% Перевод заголовков в листингах
\renewcommand\lstlistingname{Листинг}
\renewcommand\lstlistlistingname{Листинги}
% Отступ перед подписью листинга
\DeclareCaptionFormat{listing}{
    % TODO: try to use something like \indent instead of hardcoded value
    \hspace{15pt}#1#2#3
}
\captionsetup[lstlisting]{
  format=listing,
  margin=0pt,
  font={bf}
}
\lstset{
mathescape=true,
identifierstyle=\ttfamily,
keywordstyle=\bfseries,
commentstyle=\scriptsize\rmfamily,
basewidth={0.5em,0.5em},
fontadjust=true,
}
% TODO: minted vs. listings
% \usepackage[kpsewhich,newfloat]{minted}
% \usemintedstyle{vs}

%% Текст
\usepackage[inline]{enumitem} % Настройка списков, а так же "строчные" списки
% \setitemize{noitemsep,topsep=0pt,parsep=0pt,partopsep=0pt}
% \setenumerate{noitemsep,topsep=0pt,parsep=0pt,partopsep=0pt}

\usepackage[useregional]{datetime2} % Форматирование дат
\usepackage[normalem]{ulem} % Дополнительные варианты форматирования текста, например подчеркивание или зачеркивание
\usepackage{microtype} % Полезные типографические ништячки, по-хорошему требует LuaLaTeX

% TODO: Do we really need these packages?
% \usepackage{ragged2e}
% \usepackage{csquotes}

%% Разное
\usepackage{afterpage} % Выполнение команд после разрыва страниц
\usepackage{pdflscape} % Правильное отображение альбомной страницы в pdf файле

%%% Команды
% Смена \phi и \varphi
% https://tex.stackexchange.com/a/50365/171947
\expandafter\mathchardef\expandafter\varphi\number\expandafter\phi\expandafter\relax
\expandafter\mathchardef\expandafter\phi\number\varphi

% TODO: Can we get rid of this?
%https://tex.stackexchange.com/questions/30720/footnote-without-a-marker
\newcommand\blfootnote[1]{%
	\begingroup
	\renewcommand\thefootnote{}\footnote{#1}%
	\addtocounter{footnote}{-1}%
	\endgroup
}

\fancypagestyle{withCompileDate}{%
  \fancyhf{}
  \fancyfoot[C]{\thepage}
  \fancyfoot[R]{\small Дата сборки: \today}
  \renewcommand{\headrulewidth}{0pt}
}
