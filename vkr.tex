% По умолчанию используется шрифт 14 размера. Если нужен 12-й шрифт, уберите опцию [14pt]
\documentclass[14pt
  , russian
  %, xcolor={svgnames}
  ]{matmex-diploma-custom}
\usepackage[table]{xcolor}
\usepackage{graphicx}
\usepackage{tabularx}
\newcolumntype{Y}{>{\centering\arraybackslash}X}
\usepackage{amsmath}
\usepackage{amsthm}
\usepackage{amsfonts}
\usepackage{amssymb}
\usepackage{mathtools}
\usepackage{thmtools}
\usepackage{thm-restate}
\usepackage{tikz}
\usepackage{wrapfig}
% \usepackage[kpsewhich,newfloat]{minted}
% \usemintedstyle{vs}
\usepackage[inline]{enumitem}
\usepackage{subcaption}
\usepackage{caption}
%\usepackage[nocompress]{cite}
\usepackage{makecell}
% \setitemize{noitemsep,topsep=0pt,parsep=0pt,partopsep=0pt}
% \setenumerate{noitemsep,topsep=0pt,parsep=0pt,partopsep=0pt}


\graphicspath{ {resources/} }

% 
% % \documentclass 
% %   [ a4paper        % (Predefined, but who knows...)
% %   , draft,         % Show bad things.
% %   , 12pt           % Font size.
% %   , pagesize,      % Writes the paper size at special areas in DVI or
% %                    % PDF file. Recommended for use.
% %   , parskip=half   % Paragraphs: noindent + gap.
% %   , numbers=enddot % Pointed numbers.
% %   , BCOR=5mm       % Binding size correction.
% %   , submission
% %   , copyright
% %   , creativecommons 
% %   ]{eptcs}
% % \providecommand{\event}{ML 2018}  % Name of the event you are submitting to
% % \usepackage{breakurl}             % Not needed if you use pdflatex only.
% 
% \usepackage{underscore}           % Only needed if you use pdflatex.
% 
% \usepackage{booktabs}
% \usepackage{amssymb}
% \usepackage{amsmath}
% \usepackage{mathrsfs}
% \usepackage{mathtools}
% \usepackage{multirow}
% \usepackage{indentfirst}
% \usepackage{verbatim}
% \usepackage{amsmath, amssymb}
% \usepackage{graphicx}
% \usepackage{xcolor}
% \usepackage{url}
% \usepackage{stmaryrd}
% \usepackage{xspace}
% \usepackage{comment}
% \usepackage{wrapfig}
% \usepackage[caption=false]{subfig}
% \usepackage{placeins}
% \usepackage{tabularx}
% \usepackage{ragged2e}
% \usepackage{soul}
\usepackage{csquotes}
% \usepackage{inconsolata}
% 
% \usepackage{polyglossia}   % Babel replacement for XeTeX
%   \setdefaultlanguage[spelling=modern]{russian}
%   \setotherlanguage{english}
% \usepackage{fontspec}    % Provides an automatic and unified interface 
%                          % for loading fonts.
% \usepackage{xunicode}    % Generate Unicode chars from accented glyphs.
% \usepackage{xltxtra}     % "Extras" for LaTeX users of XeTeX.
% \usepackage{xecyr}       % Help with Russian.
% 
% %% Fonts
% \defaultfontfeatures{Mapping=tex-text}
% \setmainfont{CMU Serif}
% \setsansfont{CMU Sans Serif}
% \setmonofont{CMU Typewriter Text}

\usepackage[final]{listings}

\lstdefinelanguage{ocaml}{
keywords={@type, function, fun, let, in, match, with, when, class, type,
nonrec, object, method, of, rec, repeat, until, while, not, do, done, as, val, inherit, and,
new, module, sig, deriving, datatype, struct, if, then, else, open, private, virtual, include, success, failure,
lazy, assert, true, false, end},
sensitive=true,
commentstyle=\small\itshape\ttfamily,
keywordstyle=\ttfamily\bfseries, %\underbar,
identifierstyle=\ttfamily,
basewidth={0.5em,0.5em},
columns=fixed,
fontadjust=true,
literate={->}{{$\to$}}3 {===}{{$\equiv$}}1 {=/=}{{$\not\equiv$}}1 {|>}{{$\triangleright$}}3 {\\/}{{$\vee$}}2 {/\\}{{$\wedge$}}2 {>=}{{$\ge$}}1 {<=}{{$\le$}} 1,
morecomment=[s]{(*}{*)}
}

\lstset{
mathescape=true,
%basicstyle=\small,
identifierstyle=\ttfamily,
keywordstyle=\bfseries,
commentstyle=\scriptsize\rmfamily,
basewidth={0.5em,0.5em},
fontadjust=true,
language=ocaml
}
 
\newcommand{\cd}[1]{\texttt{#1}}
\newcommand{\inbr}[1]{\left<#1\right>}


\newcolumntype{L}[1]{>{\raggedright\let\newline\\\arraybackslash\hspace{0pt}}m{#1}}
\newcolumntype{C}[1]{>{\centering\let\newline\\\arraybackslash\hspace{0pt}}m{#1}}
\newcolumntype{R}[1]{>{\raggedleft\let\newline\\\arraybackslash\hspace{0pt}}m{#1}}



\usepackage{soul}
\usepackage[normalem]{ulem}
%\sout{Hello World}

% перевод заголовков в листингах
\renewcommand\lstlistingname{Листинг}
\renewcommand\lstlistlistingname{Листинги}

\usepackage{afterpage}
\usepackage{pdflscape}

% swapping \phi and \varphi
% https://tex.stackexchange.com/a/50365/171947
\expandafter\mathchardef\expandafter\varphi\number\expandafter\phi\expandafter\relax
\expandafter\mathchardef\expandafter\phi\number\varphi


\begin{document}
% Год, город, название университета и факультета предопределены,
% но можно и поменять.
% Если англоязычная титульная страница не нужна, то ее можно просто удалить.
\filltitle{ru}{
    chair              = {},
    title              = {Шаблон курсовой записки},
    % Здесь указывается тип работы. Возможные значения:
    %   coursework - Курсовая работа
    %   diploma - Отчёт по преддипломной практике
    %   master - Диплом магистра
    %   bachelor - Диплом бакалавра
    type               = {coursework},
    position           = {},
%     group              = 371,
    author             = {Типичный студент},
    supervisorPosition = {доктор физико-математических наук, профессор},
    supervisor         = {Типичный формальный научник},
    reviewerPosition   = {доктор технических наук (или программист)},
    reviewer           = {Типичный Консультант}
    % chairHeadPosition  = {д.\,ф.-м.\,н., профессор},
    % chairHead          = {Хунта К.\,Х.},
%   university         = {Санкт-Петербургский Государственный Университет},
%   faculty            = {Математико-механический факультет},
%   city               = {Санкт-Петербург},
%   year               = {2013}
}
% \filltitle{en}{
%     chair              = {Department of Software Engineering},
%     title              = {xxx},
%     author             = {xxx},
%     supervisorPosition = {xxx},
%     supervisor         = {xxxx},
%     reviewerPosition   = {Software Developer at IntelliJ Labs Co. Ltd.},
%     reviewer           = {xxx},
% }


\maketitle
\setcounter{tocdepth}{2}
\tableofcontents

% \begin{abstract}
%   В курсаче не нужен
% \end{abstract}

\section{Введение (обязателен к новому году)}

Больше инфы можно поискать в слайдах Кознова~\cite{koznov}.

Тут  4 части(абзаца) максимум на 2 страницы
\begin{enumerate}
\item Background, known information
\item Knowledge gap, unknown information
\item  Hypothesis, question, purpose statement 
\item Approach, plan of attack, proposed solution
\begin{itemize}
\item Последний абзац должен читаться и быть понятным в отрыве от остальных трёх
\end{itemize}
\end{enumerate}

\section{Постановка задачи (обязателен к новому году)}
\label{sec:task}
% !TeX spellcheck = ru_RU
 Дословно <<Целью работы является ... Для её выполнения были поставлены следующие задачи:>>
 \begin{enumerate}
 \item  реализовать это;
 \item  спроектировать это;
 \item  протестить на том-то;
 \item \sout{изучить язык Java} писать тут не надо, так как тут должны быть задачи, выполнение которых можно проверить/оценить прочитав текст или выслушав доклад.
 \end{enumerate}


\section{Обзор (обязателен к новому году)}
\label{sec:relatedworks}


\section{Background (опционально)}
TODO

\section{Метод (реализация в широком смысле)}
Что таки было сделано. Скорее всего самый большой раздел.

\emph{Крайне желательно} к Новому году иметь что-то, что сюда можно написать.

\section{Эксперимент}
Как мы проверяем, что  всё удачно получилось

\subsection{Условия эксперимента}
Железо (если актуально); входные данные, на которых проверяем наш подход; почему мы выбрали именно эти тесты

\subsection{Исследовательские вопросы (Research questions)}
Надо сформулировать то, чего мы хотели бы добиться работой (2 штуки будет хорошо):

\begin{itemize}
\item Хотим алгоритм, который лучше вот таких-то остальных
\item Если в подходе можно включать/выключать составляющие, то насколько существенно каждая составляющая влияет на улучшения
\item Если у нас строится приближение каких-то штук, то на сколько точными будут эти приближения
\item и т.п.
\end{itemize}

\subsection{Метрики}

Как мы сравниваем, что результаты двух подходов лучше или хуже
\begin{itemize}
\item Производительность
\item Строчки кода
\item Как часто алгоритм "угадывает" правильную классификацию входа
\end{itemize}

Иногда метрики вырожденные (да/нет), это не очень хорошо, но если в области исследований так принято, то ладно.

\subsection{Результаты}
Результаты понятно что такое. Тут всякие таблицы и графики

В этом разделе надо также коснуться Research Questions.

\subsubsection{RQ1} Пояснения
\subsubsection{RQ2} Пояснения

\subsection{Обсуждение результатов}

Чуть более неформальное обсуждение, то, что сделано. Например, почему метод работает лучше остальных? Или, что делать со случаями, когда метод классифицирует вход некорректно.

\section{Применение того, что сделано на практике (опциональный)}

Если применение в лоб не работает, потому что всё изложено чуть более сжато и теоретично, надо рассказать тонкости и правильный метод применения результатов. 

\section{Угрозы нарушения корректности (опциональный)}

Если основная заслуга метода, это то, что он дает лучшие цифры, то стоит сказать, где мы могли облажаться, когда проводили численные замеры. 

\section{Заключение}

Кратко, что было сделано. Также здесь стоит писать задачи на будущее.

\textbf{Для курсовых/дипломов.} Также стоит сделать список результатов, который будет 1 к одному соответствовать задачам из раздела~\ref{sec:task}.

\begin{itemize}
\item Результат к задаче 1 
\item Результат к задаче 2
\item и т.д.
\end{itemize}



% \nocite{*}
\setmonofont[Mapping=tex-text]{CMU Typewriter Text}
\bibliographystyle{ugost2008ls}
\bibliography{vkr}
\end{document}
